% The word "Abstract" should be centered 2 inches below the top of 
% the page. Then, skip one line, then center your name followed by 
% the title of your thesis/dissertation. Use as many lines as necessary.
% Centered below the title include the phrase, in parentheses,
% "(Under the direction of _________)" and include the name(s) of 
% the members of your dissertation committee.
%
% Skip one line, then begin the content of the abstract. It should be 
% double-spaced and conform to margin guidelines. An abstract should not 
% exceed 150 words for a thesis and 350 words for a dissertation. The 
% latter is a requirement of both the Graduate School and UMI's 
% Dissertation Abstracts International.
%
% Because your dissertation abstract will be published, please prepare and 
% proofread it carefully. Print all symbols and foreign words clearly and 
% accurately to avoid errors or delays. Make sure that the title given at 
% the top of the abstract has the same wording as the title shown on your 
% title page. Avoid mathematical formulas, diagrams, and other 
% illustrative materials, and only offer the briefest possible description 
% of your thesis/dissertation and a concise summary of its conclusions. Do 
% not include lengthy explanations and opinions.
%
% The abstract should bear the lower case Roman number ii (if you did not 
% include a copyright page) or iii (if you include a copyright page).

\begin{center}
\vspace*{1in} % Add 1 additional inch to 1 in. top margin
{\normalfont\textbf{ABSTRACT}}
\vspace{1em}

% Header must include your name in all caps, then the title of your
% dissertation/thesis in title font (not in all caps). Both of these
% elements must be center-aligned and single-spaced.
\begin{singlespace}
\MakeUppercase{\docAuthorFull}: \docTitle\\
(Under the direction of Advisor A. Name)
\end{singlespace}
\end{center}

% The rest of the text can be double-spaced and left-aligned.
Text

\clearpage
